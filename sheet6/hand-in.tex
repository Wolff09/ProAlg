\documentclass[11pt, oneside]{article}   	% use "amsart" instead of "article" for AMSLaTeX format
\usepackage{geometry}                		% See geometry.pdf to learn the layout options. There are lots.
\geometry{letterpaper}                   		% ... or a4paper or a5paper or ... 
%\geometry{landscape}                		% Activate for for rotated page geometry
%\usepackage[parfill]{parskip}    		% Activate to begin paragraphs with an empty line rather than an indent
\usepackage{graphicx}				% Use pdf, png, jpg, or eps§ with pdflatex; use eps in DVI mode
								% TeX will automatically convert eps --> pdf in pdflatex		
\usepackage{amssymb}

\title{Probability and Algorithms - Sheet 6}
\author{Sebastian Wolff}
\date{}							% Activate to display a given date or no date

\begin{document}
\maketitle
%\section{}
%\subsection{}

\section*{Exercise 14i}

We set $a_{ij} = \frac{w_{ij}}{2}$ and $b_{ij} = \frac{w'_{ij}}{2}$, so we get:
$\underbrace{fads}_{fdas}$

\begin{eqnarray*}
&max \sum\limits_{0 \leq i < j \leq n} a_{ij}(1-y_iy_j) + b_{ij}(1+y_iy_j) \\
&= max \sum\limits_{0 \leq i < j \leq n} a_{ij}(1-y_iy_j) + \sum\limits_{0 \leq i < j \leq n} b_{ij}(1+y_iy_j) \\
&= max \underbrace{\sum\limits_{0 \leq i < j \leq n} w_{ij}\frac{(1-y_iy_j)}{2}}_{ \textrm{max cut with n+1 nodes} } + \underbrace{\sum\limits_{0 \leq i < j \leq n} w'_{ij}\frac{(1+y_iy_j)}{2}}_{\textrm{min cut with n+1 nodes}}
\end{eqnarray*}

Hence, we need to encode MAX2SAT as a maximum of a MAX-CUT and a MIN-CUT of n+1 nodes (the MAX-CUT and the MIN-CUT can be computed on different graphs).
Note that the MAX2SAT can be solved with a MAX-CUT with 2n nodes on implication graphs.

\textbf{Idea:} use the node for $y_0$ to encode some kind of supersink/supersource to connect both graphs.


\section*{Exercise 14ii}

Define $\tilde{K}$ and $\hat{K}$ as follows:

\begin{eqnarray*}
\tilde{K}(V) = \sum\limits_{0 \leq i < j \leq n} a_{ij}\frac{arccos(v_i^Tv_j)}{\pi}
\hat{K}(V) = \sum\limits_{0 \leq i < j \leq n} b_{ij}(1- \frac{arccos(v_i^Tv_j)}{\pi})
\end{eqnarray*}

Since $E(V) = \tilde{K}(V) + \hat{K}(V)$ and we know from lecture that
\begin{eqnarray*}
\tilde{K}(V) = \sum\limits_{0 \leq i < j \leq n} a_{ij}(1-v_i^Tv_j)
\end{eqnarray*}
holds, it remains to show:
\begin{eqnarray*}
\hat{K}(V) = \sum\limits_{0 \leq i < j \leq n} b_{ij}(1+v_i^Tv_j)
\end{eqnarray*}
So here we go:
\begin{eqnarray*}
&\sum\limits_{0 \leq i < j \leq n} b_{ij}(1+v_i^Tv_j) \\
&=\sum\limits_{0 \leq i < j \leq n} b_{ij}Pr(sign(v_i^Tr)=sign(v_j^Tr)) \\
&=\sum\limits_{0 \leq i < j \leq n} b_{ij}(1-Pr(sign(v_i^Tr)\neq sign(v_j^Tr))) \\
&=\sum\limits_{0 \leq i < j \leq n} b_{ij}(1-2Pr(v_i^Tr\leq 0 \wedge v_j^Tr > 0)) \\
&=\sum\limits_{0 \leq i < j \leq n} b_{ij}(1-2\frac{\theta}{2\pi}) \\
&=\sum\limits_{0 \leq i < j \leq n} b_{ij}(1-\frac{\theta}{\pi}) \\
&=\sum\limits_{0 \leq i < j \leq n} b_{ij}(1-\frac{arccos(v_i^Tv_j)}{\pi}) \\
&=\hat{K}(V) \\
%&=\sum\limits_{0 \leq i < j \leq n} b_{ij} - \sum\limits_{0 \leq i < j \leq n} \frac{arccos(v_i^Tv_j}{\pi} \\
%&=\sum\limits_{0 \leq i < j \leq n} b_{ij} - \tilde{K}(V)
\end{eqnarray*}
When we put things together we get:
\begin{eqnarray*}
E(V) = \tilde{K}(V) + \hat{K}(V) = \sum\limits_{0 \leq i < j \leq n} a_{ij}(1-v_i^Tv_j)+b_{ij}(1+v_i^Tv_j)
\end{eqnarray*}
Together with $\alpha \leq 1$ we get:
\begin{eqnarray*}
E(V) = \alpha(\tilde{K}(V) + \hat{K}(V)) = \alpha(\sum\limits_{0 \leq i < j \leq n} a_{ij}(1-v_i^Tv_j)+b_{ij}(1+v_i^Tv_j))
\end{eqnarray*}


\end{document}